\subsection{Functional Requirements}
\paragraph{Trip Recording \& Statistics}
\begin{description}
    \item[\textbf{R1 - User registration}] 
    The system shall allow a user to create an account using an email (or federated login) and password.

    \item[\textbf{R2 - User login}] 
    The system shall allow a registered user to authenticate and access their personal data.

    \item[\textbf{R3 - Start trip recording}] 
    The system shall allow a logged-in user to start recording a new trip, initializing GPS logging and timing.

    \item[\textbf{R4 - Stop trip recording}] 
    The system shall allow the user to stop an ongoing trip and save it as a completed trip.

    \item[\textbf{R5 - Trip statistics}] 
    After a trip is saved, the system shall compute and store at least distance, duration and average speed.

    \item[\textbf{R6 - Trip history}] 
    The system shall allow a logged-in user to view the list of all recorded trips and open detailed views.

    \item[\textbf{R7 - Trip weather enrichment}] 
    For each trip, if the weather service is reachable, the system shall fetch and attach weather information based on time and location.
\end{description}

\paragraph{Manual Path Information}
\begin{description}
    \item[\textbf{R8 - Manual path creation}] 
    The system shall allow a logged-in user to select or draw a bike path using the map or by specifying street names.

    \item[\textbf{R9 - Segment status assignment}] 
    The system shall allow the user to set a status (optimal, medium, sufficient, requires maintenance, etc.) for each segment of a path.

    \item[\textbf{R10 - Obstacle creation}] 
    The system shall allow the user to create obstacles associated with a segment, specifying the obstacle type and a description.

    \item[\textbf{R11 - Publishability}] 
    The system shall allow the user to mark manual contributions as publishable or private.

    \item[\textbf{R12 - Edit manual contributions}] 
    The system shall allow users to edit or delete their previously submitted path reports.
\end{description}

\paragraph{Automatic path information}
\begin{description}
    \item[\textbf{R13 - Enable/disable automatic mode}] 
    The system shall allow a user to enable or disable automatic collection of sensor data for path quality detection.

    \item[\textbf{R14 - Sensor data logging}] 
    When automatic mode is enabled and a trip is being recorded, the system shall periodically sample accelerometer and gyroscope data and associate them with geographic positions.

    \item[\textbf{R15 - Detection of candidate obstacles}] 
    The system shall process sensor data and detect significant events (e.g., strong jolts) as candidate potholes or rough path segments.

    \item[\textbf{R16 - Candidate list presentation}] 
    After a trip ends, the system shall present the user with a list or map of candidate issues detected during that trip.

    \item[\textbf{R17 - User confirmation/correction}] 
    For each candidate, the user shall be able to confirm it, reject it, or edit its details (type, severity, or position) before it becomes a report.

    \item[\textbf{R18 - Conversion into reports}] 
    Confirmed candidates shall be stored as \textit{PathSegmentReport} objects associated with path segments. Rejected candidates shall not be stored as publishable data.
\end{description}

\paragraph{ Path search and visualization}
\begin{description}
    \item[\textbf{R19 - Public path search}] 
    The system shall allow any user, whether registered or not, to specify an origin and destination and request bike paths.

    \item[\textbf{R20 - Route computation}] 
    The system shall compute one or more route candidates using external map and routing services in combination with the internal path database.

    \item[\textbf{R21 - Path scoring}] 
    The system shall compute a score for each candidate path based on:
    \begin{itemize}
        \item Consolidated statuses of included segments.
        \item Severity and number of obstacles.
        \item Route effectiveness (e.g., distance, elevation, number of turns).
    \end{itemize} 
    \item[\textbf{R22 - Ordered path list}] 
    The system shall present candidate paths ordered by decreasing score.

    \item[\textbf{R23 - Path visualization}] 
    The system shall visualize a selected path on a map with overlays representing segment status and reported obstacles.
\end{description}

\paragraph{Data Merging}
\begin{description}
    \item[\textbf{R24 - Segment report storage}] 
    The system shall store all PathSegmentReports with timestamps and user identifiers.

    \item[\textbf{R25 - Periodic merging}] 
    The system shall periodically recompute the consolidated status of each segment based on all available reports.

    \item[\textbf{R26 - Freshness handling}] 
    The merging process shall weigh newer reports more heavily than older ones.

    \item[\textbf{R27 - Majority handling}] 
    If multiple reports with similar freshness disagree, the consolidated status shall follow the majority assessment (e.g., by count or weighted count).
\end{description}

\paragraph{Administration and data quality}
\begin{description}
    \item[\textbf{R28 - User Blocking}] 
    The system shall allow administrators to block users who repeatedly submit obviously false data.

    \item[\textbf{R29 - Data Removal}] 
    Administrators shall be able to remove or hide problematic reports.
\end{description}


\newpage

% Ragged-right p-columns (prevents most Underfull \hbox in tables)
\newcolumntype{P}[1]{>{\RaggedRight\arraybackslash}p{#1}}

% --- Your table (rectified) ---
\subsection{Requirements Traceability Table}

\renewcommand{\arraystretch}{1.15}
\setlength{\tabcolsep}{4pt}

% optional: helps TeX avoid both underfull/overfull in tight tables
\begingroup
\sloppy
\setlength{\emergencystretch}{1.5em}

\begin{longtable}{|P{1.1cm}|P{3.3cm}|P{2.2cm}|P{3.0cm}|P{5.1cm}|}
\caption{Requirements traceability table}
\label{tab:req-traceability}\\
\hline
\textbf{Req ID} & \textbf{Requirement (short)} & \textbf{Flutter Module(s)} & \textbf{Firebase Services} & \textbf{Main Data/Artifacts} \\
\hline
\endfirsthead

\hline
\textbf{Req ID} & \textbf{Requirement (short)} & \textbf{Flutter Module(s)} & \textbf{Firebase Services} & \textbf{Main Data/Artifacts} \\
\hline
\endhead



\endlastfoot

R1  & User registration & A1 & F1, F2 & \texttt{users/\{uid\}} \\
\hline
R2  & User login & A1 & F1, F2 & \texttt{users/\{uid\}} \\
\hline
R3  & Start trip recording & A2 & (F2 optional) & local GPS buffer \\
\hline
R4  & Stop trip recording & A2, A3 & F2, F3, (F4 optional) &
\texttt{trips/\allowbreak\{tripId\}} + trace file \\
\hline
R5  & Trip statistics & A2, A3 & F2, (F4 optional) & fields in \texttt{trips/\{tripId\}} \\
\hline
R6  & Trip history & A3 & F2 & query trips by uid \\
\hline
R7  & Trip weather enrichment & A3 & F4, F2 & \texttt{trips.weatherSnapshot} \\
\hline
R8  & Manual path creation & A4 & F2 & pathSegments, pathSegmentReports \\
\hline
R9  & Segment status assignment & A4 & F2 &
\seqsplit{pathSegmentReports.segmentStatus} \\
\hline
R10 & Obstacle creation & A4 & F2, (F3 optional) & obstacles (optional photo in Storage) \\
\hline
R11 & Publishability & A4 & F2 & publishable flag on reports and obstacles \\
\hline
R12 & Edit manual contributions & A4 & F2, (F4 validation optional) & update or delete own pathSegmentReports \\
\hline
R13 & Enable/disable automatic mode & A1, A2, A5 & F2 & \texttt{users.automaticModeEnabled} \\
\hline
R14 & Sensor data logging & A2 & (F2 optional), (F3 optional) & sensor samples + GPS tags \\
\hline
R15 & Detect candidate obstacles & A5 & (F4 optional) & candidate events (local or stored) \\
\hline
R16 & Candidate list presentation & A5 & none required & candidate UI list or map \\
\hline
R17 & User confirmation or correction & A5 & F2 & confirmed or edited candidate data \\
\hline
R18 & Conversion into reports & A5 & F2 & create pathSegmentReports and obstacles \\
\hline
R19 & Public path search & A6 & F2, F4 & read consolidated data + compute route \\
\hline
R20 & Route computation & A6 & F4 + external routing &
\seqsplit{computeRoutesAndScores()} + MapService \\
\hline
R21 & Path scoring & A6 & F4, F2 & uses consolidatedStatuses + obstacles \\
\hline
R22 & Ordered path list & A6 & F4 & sorted candidates returned \\
\hline
R23 & Path visualization & A6 & F2 & overlays from consolidated statuses + obstacles \\
\hline
R24 & Segment report storage & A4, A5 & F2 & pathSegmentReports with uid + timestamps \\
\hline
R25 & Periodic merging & none & F5, F4, F2 & refresh consolidatedStatuses \\
\hline
R26 & Freshness handling & none & F4, F2 & timestamp weighting in merge \\
\hline
R27 & Majority handling & none & F4, F2 & majority or weighted-majority merge rule \\
\hline
R28 & User blocking & A7 & F1 (role), F2, F4 & users.blocked + admin enforcement \\
\hline
R29 & Data removal or hide & A7 & F2, F4 & hide or delete report + recompute merge \\
\hline

\end{longtable}
\endgroup
