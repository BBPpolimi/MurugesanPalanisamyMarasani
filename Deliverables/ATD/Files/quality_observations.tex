\subsection{Code Readability}
\begin{itemize}
    \item Clear package separation: \texttt{handlers/}, \texttt{models/}, \texttt{database/}, \texttt{weather/}, \texttt{static/}
    \item Descriptive naming for functions and variables with self explanatory intent
    \item Consistent use of Go idioms such as explicit \texttt{err} handling concise handlers JSON tags and DTO usage
    \item Authentication utilities are clean and readable including \texttt{generateToken}, \texttt{hashPassword} and \texttt{checkPassword}
    \item Useful inline comments for key business logic and requirement traceability for example \texttt{// (R16)}
    \item SQL schema is well commented and easy to understand
    \item Inconsistent error handling as some errors are logged with fallback behavior while others are silently ignored
    \item Long and complex functions in \texttt{handlers/trips.go} reduce clarity due to mixed responsibilities
    \item Repeated SQL NULL parsing patterns introduce duplication and reduce maintainability
    \item Magic numbers and hardcoded constants such as session duration and bcrypt cost are present without centralized configuration
    \item Minor robustness and security related readability issues such as raw SQL \texttt{LIMIT} concatenation
\end{itemize}

\subsection{Project Structure}
\begin{itemize}
    \item Clean repository level organization with separate documentation directories such as \texttt{RASD/}, \texttt{DD/} and \texttt{Implementation/}
    \item Implementation structure is conventional and easy to navigate
    \item Entry point is simple and contained in \texttt{main.go}
    \item Docker and Docker Compose configuration is provided and the use of a multi stage Docker build is a strong design choice
    \item Database schema is embedded and managed cleanly with volume persistence properly configured
    \item External dependencies are minimal and appropriate
    \item No explicit service layer is present as business logic is embedded directly in HTTP handlers
    \item No repository or data access abstraction layer is implemented and SQL queries are mixed into handlers
    \item Middleware pipeline is limited with no centralized logging metrics or recovery mechanism
    \item No automated tests are provided as \texttt{*\_test.go} files are missing across packages
    \item Potential evaluator setup friction exists as the PostgreSQL port mapping may conflict with local usage on port 5432
\end{itemize}

\subsection{Documentation Clarity}
\begin{itemize}
    \item The root README provides clear setup commands and a usage entry point including the application URL
    \item The implementation README includes a useful endpoint list which supports black box evaluation
    \item Repository structure overview is clear and evaluator friendly
    \item A public deployment URL is documented and useful for verification
    \item SQL schema documentation is detailed with tables constraints and seeded locations clearly explained
    \item No troubleshooting section is provided for common issues such as port conflicts Docker reset or service startup timing
    \item Environment variable and configuration documentation is minimal for example database URL and port configuration
    \item No formal API documentation is provided such as OpenAPI Swagger or request and response examples
    \item Package level GoDoc comments are missing which reduces long term maintainability
\end{itemize}

\subsection{Coherence with RASD and DD}
\begin{itemize}
    \item Strong coverage of core RASD requirements including R1, R2, R3, R7, R8, R9, R13, R14, R15 and R16
    \item Weather enrichment functionality aligns with RASD goal G6 and the designed external service integration
    \item Publishing and community related behavior aligns with aggregation and shared knowledge goals defined in G5
    \item Search and route visualization features align with browsing and recommendation goals defined in G4
    \item Database schema is coherent with the DD graph model concept including nodes edges user reports and aggregation
    \item Implementation maps well to the component responsibilities described in the DD including authentication trips paths routing search database and weather
    \item Sensor based automatic detection goals related to G3 and requirements R4 to R6 are simulated or limited compared to real device sensing
    \item The DD layering is conceptually present using a client server model but implemented as a monolithic backend which is acceptable for a prototype
    \item A minor documented deviation is present where nginx is used instead of ngrok which represents an implementation choice and does not violate requirements
\end{itemize}
