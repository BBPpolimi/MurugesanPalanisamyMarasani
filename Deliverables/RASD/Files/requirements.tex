\subsection{External Interface Requirements}

\subsubsection{User Interfaces}
\begin{enumerate}
    \item \textbf{Home/Dashboard (registered user):}
    \begin{itemize}
        \item Shows quick buttons: \textit{Start Trip}, \textit{My Trips}, \textit{Contribute Paths}, \textit{Find Path}. Shows recent trips and short stats.
    \begin{figure}[H]
        \centering
        \includegraphics[width=0.5\textwidth]{Images/figma_diagrams/Home_Dashboard.png}
        \caption{\label{fig:figma_home_dashboard}Dashboard}
    \end{figure}
    \vspace{1cm}
    \end{itemize}
    \item \textbf{Trip Recording Screen:} 
    \begin{itemize}
        \item Displays elapsed time, distance and current speed.
        \item Shows the user’s position on a map.
        \item Contains buttons: Start, Pause/Resume, Stop.
        \begin{figure}[H]
        \centering
        \includegraphics[width=0.5\textwidth]{Images/figma_diagrams/Trip Recording.png}
        \caption{\label{fig:figma_trip_record}Trip Recording Screen}
    \end{figure}
    \end{itemize}
    \item \textbf{Trip Details Screen:}
    \begin{itemize}
        \item Map with route polyline.
        \item Trip statistics.
        \item Weather summary.
        \item Optional toggle “use this trip to update path information”.
        \begin{figure}[H]
        \centering
        \includegraphics[width=0.5\textwidth]{Images/figma_diagrams/Trip Details (Post-Trip).png}
        \caption{\label{fig:figma_trip_details}Trip Details Screen}
    \end{figure}
    \end{itemize}
    \item \textbf{Contribute Paths (Manual) Screen}
    \begin{itemize}
        \item Map where the user can draw/select a path or list of streets.
        \item Form for assigning status to each segment and adding obstacles.
        \item Checkbox or switch \textit{Make this contribution publishable}.
        \begin{figure}[H]
        \centering
        \includegraphics[width=0.5\textwidth]{Images/figma_diagrams/Contribute Paths (Manual).png}
        \caption{\label{fig:figma_contribute_path_manual}Contribute Path (Manual) Screen}
    \end{figure}
    \end{itemize}
    \item \textbf{Automatic Detection Review Screen}
    \begin{itemize}
        \item List or map of suspected obstacles detected during recent trips.
        \item For each candidate: buttons \textit{Confirm, Reject, Edit details.}
        \begin{figure}[H]
        \centering
        \includegraphics[width=0.5\textwidth]{Images/figma_diagrams/Automatic Detection Review.png}
        \caption{\label{fig:figma_automatic_detection_review}Automatic Detection Review}
    \end{figure}

    \end{itemize}
    \item \textbf{Path Search Screen (Public)}
    \begin{itemize}
        \item Input fields for origin and destination (address, point on map).
        \item Optional filters (e.g., \textit{prefer safest, shortest path, avoid unlit segments}).
        \item List of candidate paths with score and summary (distance, expected quality).
        \item Map visualization of selected path.
        \begin{figure}[H]
        \centering
        \includegraphics[width=0.45\textwidth]{Images/figma_diagrams/Path Search.png}
        \caption{\label{fig:figma_path_search}Path Search Screen}
    \end{figure}
    \end{itemize}

\end{enumerate}

\subsubsection{Hardware Interfaces}
\begin{enumerate}
    \item Device with:
    \begin{itemize}
        \item GPS Receiver
        \item Accelerometer and gyroscope (for automatic mode).
        \item Network connectivity.
    \end{itemize}
    \item BBP accesses these sensors through the OS APIs (e.g.android/iOS location \& sensor APIs).
\end{enumerate}

\subsubsection{Software Interfaces}
\begin{enumerate}
    \item \textbf{Map/Routing API:}  For map tiles, geocoding (address <==> coordinates)and route computation.
    \item \textbf{Weather API:} BBP calls the external weather service with time and location to enrich trips with weather snapshots.
    \item \textbf{Persistent Storage (DB):} For users, trips, segments, reports, merged statuses.
\end{enumerate}
\subsubsection{ Communication Interfaces}
\begin{enumerate}
    \item All communication between client and server uses HTTPS.
    \item Calls to external APIs (maps, weather) also use secure HTTP endpoints.
    \item Where live updates are needed (e.g., showing trip progress on web), WebSockets or equivalent may be used.
\end{enumerate}

\subsection{Functional Requirements}
\paragraph{Trip Recording \& Statistics}
\begin{description}
    \item[\textbf{R1 - User registration}] 
    The system shall allow a user to create an account using an email (or federated login) and password.

    \item[\textbf{R2 - User login}] 
    The system shall allow a registered user to authenticate and access their personal data.

    \item[\textbf{R3 - Start trip recording}] 
    The system shall allow a logged-in user to start recording a new trip, initializing GPS logging and timing.

    \item[\textbf{R4 - Stop trip recording}] 
    The system shall allow the user to stop an ongoing trip and save it as a completed trip.

    \item[\textbf{R5 - Trip statistics}] 
    After a trip is saved, the system shall compute and store at least distance, duration, and average speed.

    \item[\textbf{R6 - Trip history}] 
    The system shall allow a logged-in user to view the list of all recorded trips and open detailed views.

    \item[\textbf{R7 - Trip weather enrichment}] 
    For each trip, if the weather service is reachable, the system shall fetch and attach weather information based on time and location.
\end{description}

\paragraph{Manual Path Information}
\begin{description}
    \item[\textbf{R8 - Manual path creation}] 
    The system shall allow a logged-in user to select or draw a bike path using the map or by specifying street names.

    \item[\textbf{R9 - Segment status assignment}] 
    The system shall allow the user to set a status (optimal, medium, sufficient, requires maintenance, etc.) for each segment of a path.

    \item[\textbf{R10 - Obstacle creation}] 
    The system shall allow the user to create obstacles associated with a segment, specifying the obstacle type and a description.

    \item[\textbf{R11 - Publishability}] 
    The system shall allow the user to mark manual contributions as publishable or private.

    \item[\textbf{R12 - Edit manual contributions}] 
    The system shall allow users to edit or delete their previously submitted path reports.
\end{description}

\paragraph{Automatic path information}
\begin{description}
    \item[\textbf{R13 - Enable/disable automatic mode}] 
    The system shall allow a user to enable or disable automatic collection of sensor data for path quality detection.

    \item[\textbf{R14 - Sensor data logging}] 
    When automatic mode is enabled and a trip is being recorded, the system shall periodically sample accelerometer and gyroscope data and associate them with geographic positions.

    \item[\textbf{R15 - Detection of candidate obstacles}] 
    The system shall process sensor data and detect significant events (e.g., strong jolts) as candidate potholes or rough path segments.

    \item[\textbf{R16 - Candidate list presentation}] 
    After a trip ends, the system shall present the user with a list or map of candidate issues detected during that trip.

    \item[\textbf{R17 - User confirmation/correction}] 
    For each candidate, the user shall be able to confirm it, reject it, or edit its details (type, severity, or position) before it becomes a report.

    \item[\textbf{R18 - Conversion into reports}] 
    Confirmed candidates shall be stored as \textit{PathSegmentReport} objects associated with path segments. Rejected candidates shall not be stored as publishable data.
\end{description}

\paragraph{ Path search and visualization}
\begin{description}
    \item[\textbf{R19 - Public path search}] 
    The system shall allow any user, whether registered or not, to specify an origin and destination and request bike paths.

    \item[\textbf{R20 - Route computation}] 
    The system shall compute one or more route candidates using external map and routing services in combination with the internal path database.

    \item[\textbf{R21 - Path scoring}] 
    The system shall compute a score for each candidate path based on:
    \begin{itemize}
        \item Consolidated statuses of included segments.
        \item Severity and number of obstacles.
        \item Route effectiveness (e.g., distance, elevation, number of turns).
    \end{itemize} 
    \item[\textbf{R22 - Ordered path list}] 
    The system shall present candidate paths ordered by decreasing score.

    \item[\textbf{R23 - Path visualization}] 
    The system shall visualize a selected path on a map with overlays representing segment status and reported obstacles.
\end{description}

\paragraph{Data Merging}
\begin{description}
    \item[\textbf{R24 - Segment report storage}] 
    The system shall store all PathSegmentReports with timestamps and user identifiers.

    \item[\textbf{R25 - Periodic merging}] 
    The system shall periodically recompute the consolidated status of each segment based on all available reports.

    \item[\textbf{R26 - Freshness handling}] 
    The merging process shall weigh newer reports more heavily than older ones.

    \item[\textbf{R27 - Majority handling}] 
    If multiple reports with similar freshness disagree, the consolidated status shall follow the majority assessment (e.g., by count or weighted count).
\end{description}

\paragraph{Administration and data quality}
\begin{description}
    \item[\textbf{R28 - User Blocking}] 
    The system shall allow administrators to block users who repeatedly submit obviously false data.

    \item[\textbf{R29 - Data Removal}] 
    Administrators shall be able to remove or hide problematic reports.
\end{description}

\subsubsection{Use Case Diagrams}
The diagrams that from Fig. ~\ref{fig:use_case_dig1} to Fig. ~\ref{fig:use_case_dig4} are the use case diagrams shown below:
\begin{figure}[H]
\centering
\includegraphics[width=\textwidth]{Images/use_case_diagrams/Diagram_1_Account_Management _Trip_Recording.jpg}
\caption{\label{fig:use_case_dig1}Use Case Diagram for Account Management.}
\end{figure}



\begin{figure}[H]
\centering
\includegraphics[width=\textwidth]{Images/use_case_diagrams/Diagram 2 – Manual Path Contribution.jpg}
\caption{\label{fig:use_case_dig2}Use Case Diagram for Manual path Contribution.}
\end{figure}


\begin{figure}[H]
\centering
\includegraphics[width=\textwidth]{Images/use_case_diagrams/Diagram 3 – Automatic Path Information & Confirmation.jpg}
\caption{\label{fig:use_case_dig3}Use Case Diagram for Automatci Path Information and Confirmation.}
\end{figure}




\begin{figure}[H]
\centering
\includegraphics[width=\textwidth]{Images/use_case_diagrams/Diagram 4 – Path Search, Scoring & Merging.jpg}
\caption{\label{fig:use_case_dig4}Use Case Diagram for Path Search, Scoring and Merging.}
\end{figure}


\newpage
\subsubsection{Use Cases}
The use cases of Best Bike Path (BBP) are shown as Table.~\ref{table:use_case_table1} to Table.~\ref{table:use_case_table7}

\textbf{UC1: User Registration}
\begin{table}[H]
\centering
\begin{tabular}{|p{3.5cm}|p{10.5cm}|}
\hline
\textbf{Name} & \textbf{User Registration} \\
\hline
\textbf{Actors} & Visitor \\
\hline
\textbf{Entry Condition} & Visitor opens the BBP application. \\
\hline
\textbf{Event Flow} &
(a) The visitor selects \textit{Sign Up}. \\[0.3em]
& (b) The system displays the registration form requesting email, password, and optional name. \\[0.3em]
& (c) The visitor enters the required information and confirms the registration. \\[0.3em]
& (d) The system validates the provided data and creates a new user account. \\[0.3em]
& (e) The system confirms successful registration and may automatically log the user in. \\
\hline
\textbf{Exit Condition} & A new user account exists. \\
\hline
\textbf{Exceptions} & Email already used; invalid email format; weak password. \\
\hline
\end{tabular}
\caption{Use Case 1: User Registration}
\label{table:use_case_table1}
\end{table}

\textbf{UC2: User Login}
\begin{table}[H]
\centering
\begin{tabular}{|p{3.5cm}|p{10.5cm}|}
\hline
\textbf{Name} & \textbf{User Login} \\
\hline
\textbf{Actors} & Registered User \\
\hline
\textbf{Entry Condition} & User opens the BBP application. \\
\hline
\textbf{Event Flow} &
(a) The user selects \textit{Login}. \\[0.3em]
& (b) The system prompts the user to enter credentials. \\[0.3em]
& (c) The user enters their email and password. \\[0.3em]
& (d) The system validates the provided credentials. \\[0.3em]
& (e) The system grants access and displays the user dashboard. \\
\hline
\textbf{Exit Condition} & The user is successfully logged in. \\
\hline
\textbf{Exceptions} & Invalid credentials; account blocked. \\
\hline
\end{tabular}
\caption{Use Case 2: User Login}
\label{table:use_case_table2}
\end{table}

\newpage

\textbf{UC3: Record Trip}
\begin{table}[H]
\centering
\begin{tabular}{|p{3.5cm}|p{10.5cm}|}
\hline
\textbf{Name} & \textbf{Record Trip} \\
\hline
\textbf{Actors} & Registered User \\
\hline
\textbf{Entry Condition} & User is logged in. \\
\hline
\textbf{Event Flow} &
(a) The user taps \textit{Start Trip}. \\[0.3em]
& (b) The system requests permission to access location data, if not already granted. \\[0.3em]
& (c) The system begins logging GPS points and the trip start time. \\[0.3em]
& (d) The user rides while the system continuously updates position and trip statistics. \\[0.3em]
& (e) The user taps \textit{Stop Trip}. \\[0.3em]
& (f) The system stops logging, computes trip statistics, and saves the completed trip. \\[0.3em]
& (g) The system requests weather data and attaches it to the trip, if available. \\
\hline
\textbf{Exit Condition} & The completed trip is stored with statistics and, if available, weather information. \\
\hline
\textbf{Exceptions} & Location permission denied; GPS unavailable; network unavailable for weather data (trip is stored without weather information). \\
\hline
\end{tabular}
\caption{Use Case 3: Record Trip}
\label{table:use_case_table3}
\end{table}

\textbf{UC4: Add Manual Path Information}
\begin{table}[H]
\centering
\begin{tabular}{|p{3.5cm}|p{10.5cm}|}
\hline
\textbf{Name} & \textbf{Add Manual Path Information} \\
\hline
\textbf{Actors} & Registered User \\
\hline
\textbf{Entry Condition} & User is logged in. \\
\hline
\textbf{Event Flow} &
(a) The user opens \textit{Contribute Paths}. \\[0.3em]
& (b) The system displays a map and tools for selecting streets or drawing a path. \\[0.3em]
& (c) The user defines one or more path segments. \\[0.3em]
& (d) For each segment, the user selects a path status. \\[0.3em]
& (e) The user optionally adds obstacles to segments, specifying type and severity. \\[0.3em]
& (f) The user chooses whether the contributed information is publishable. \\[0.3em]
& (g) The system saves the path segments and associated reports. \\
\hline
\textbf{Exit Condition} & Path segment reports are created and stored; consolidated segment status may be updated. \\
\hline
\textbf{Exceptions} & Incomplete information (e.g., no segments selected); map service unavailable. \\
\hline
\end{tabular}
\caption{Use Case 4: Add Manual Path Information}
\label{table:use_case_table4}
\end{table}

\newpage

\textbf{UC5: Automatic Detection Confirmation}
\begin{table}[H]
\centering
\begin{tabular}{|p{3.5cm}|p{10.5cm}|}
\hline
\textbf{Name} & \textbf{Automatic Detection Confirmation} \\
\hline
\textbf{Actors} & Registered User \\
\hline
\textbf{Entry Condition} & User has completed a trip with automatic mode enabled. \\
\hline
\textbf{Event Flow} &
(a) After the trip, the system computes candidate obstacles. \\[0.3em]
& (b) The system notifies the user that new candidates are available for review. \\[0.3em]
& (c) The user opens the review screen. \\[0.3em]
& (d) The system lists detected candidates with map positions and basic information. \\[0.3em]
& (e) For each candidate, the user selects \textit{Confirm}, \textit{Reject}, or \textit{Edit}. \\[0.3em]
& (f) The system converts confirmed items into reports, discards rejected candidates, and saves any edits. \\
\hline
\textbf{Exit Condition} & Confirmed obstacles are stored as publishable or pending publication according to the user’s choice. \\
\hline
\textbf{Exceptions} & User skips the review; in this case, candidate obstacles are not published. \\
\hline
\end{tabular}
\caption{Use Case 5: Automatic Detection Confirmation}
\label{table:use_case_table5}
\end{table}

\textbf{UC6: Search Path Between Origin and Destination}
\begin{table}[H]
\centering
\begin{tabular}{|p{3.5cm}|p{10.5cm}|}
\hline
\textbf{Name} & \textbf{Search Path Between Origin and Destination} \\
\hline
\textbf{Actors} & Guest User, Registered User \\
\hline
\textbf{Entry Condition} & User is on the BBP main screen. \\
\hline
\textbf{Event Flow} &
(a) The user enters an origin and destination as addresses or map points. \\[0.3em]
& (b) The user initiates the search. \\[0.3em]
& (c) The system geocodes the origin and destination. \\[0.3em]
& (d) The system computes one or more candidate routes. \\[0.3em]
& (e) For each route, the system retrieves consolidated segment statuses and reported obstacles. \\[0.3em]
& (f) The system computes scores for each route and sorts them accordingly. \\[0.3em]
& (g) The system displays a list of candidate routes and allows the user to select one. \\[0.3em]
& (h) The system displays the selected path on the map with detailed information. \\
\hline
\textbf{Exit Condition} & The user views at least one candidate path on the map. \\
\hline
\textbf{Exceptions} & Invalid addresses; no path found; external services unavailable. \\
\hline
\end{tabular}
\caption{Use Case 6: Search Path Between Origin and Destination}
\label{table:use_case_table6}
\end{table}

\newpage

\textbf{UC7: Merging Path Reports}
\begin{table}[H]
\centering
\begin{tabular}{|p{3.5cm}|p{10.5cm}|}
\hline
\textbf{Name} & \textbf{ Merging Path Reports} \\
\hline
\textbf{Actors} & Time (scheduled job), Administrator (indirect observer) \\
\hline
\textbf{Entry Condition} & The system has accumulated multiple reports for path segments. \\
\hline
\textbf{Event Flow} &
(a) At scheduled intervals, the merging procedure is triggered. \\[0.3em]
& (b) The system gathers all reports associated with each path segment. \\[0.3em]
& (c) The system groups reports by segment and sorts them by timestamp. \\[0.3em]
& (d) The system applies merging rules based on report freshness and majority. \\[0.3em]
& (e) The system updates each segment’s consolidated status and last-updated date. \\
\hline
\textbf{Exit Condition} & Consolidated segment status reflects the latest known information. \\
\hline
\textbf{Exceptions} & None specific; the merging process may be skipped if no new reports are available. \\
\hline
\end{tabular}
\caption{Use Case 7: Merging Path Reports}
\label{table:use_case_table7}
\end{table}

\newpage

\subsubsection{Sequence Diagrams}
The diagrams from Fig.~\ref{fig:sequence_dig1} to Fig.~\ref{fig:sequence_dig7} represents use cases from UC1 to UC7


\begin{figure}[H]
\centering
\includegraphics[width=\textwidth]{Images/sequence_diagrams/UC1_page-0001.jpg}
\caption{\label{fig:sequence_dig1}Sequence Diagram for UC1-User Registration.}
\end{figure}


\begin{figure}[H]
\centering
\includegraphics[width=\textwidth]{Images/sequence_diagrams/UC2_page-0001.jpg}
\caption{\label{fig:sequence_dig2}Sequence Diagram for UC2-User Login.} 
\end{figure}


\begin{figure}[H]
\centering
\includegraphics[width=0.7\textwidth]{Images/sequence_diagrams/UC3_page-0001.jpg}
\caption{\label{fig:sequence_dig3}Sequence Diagram for UC3- Record Trip}
\end{figure}


\begin{figure}[H]
\centering
\includegraphics[width=0.7\textwidth]{Images/sequence_diagrams/UC3_page-0001.jpg}
\caption{\label{fig:sequence_dig4}Sequence Diagram for UC4- Add Manual Path Information}
\end{figure}


\begin{figure}[H]
\centering
\includegraphics[width=\textwidth]{Images/sequence_diagrams/uc5_page-0001.jpg}
\caption{\label{fig:sequence_dig5}Sequence Diagram for UC5- Automatic Dectection Confirmation}
\end{figure}


\begin{figure}[H]
\centering
\includegraphics[width=0.7\textwidth]{Images/sequence_diagrams/uc6_page-0001.jpg}
\caption{\label{fig:sequence_dig6}Sequence Diagram for UC6- Search Path Between Origin and Destination}
\end{figure}


\begin{figure}[H]
\centering
\includegraphics[width=\textwidth]{Images/sequence_diagrams/uc7_page-0001.jpg}
\caption{\label{fig:sequence_dig7}Sequence Diagram for UC7- Mergin Path Reports}
\end{figure}






\subsubsection{Requirement Mapping}

% -------------------- G1 --------------------
\begin{table}[H]
\centering
\begin{tabular}{|p{7cm}|p{7cm}|}
\hline
\multicolumn{2}{|p{15cm}|}{\textbf{G1: Personal trip tracking:} Registered users can record, store, and review trips with key statistics and map visualization.} \\
\hline
\textbf{Mapped Functional Requirements} &
\textbf{Mapped Domain Assumptions / Dependencies} \\
\hline
R1 -- User registration \newline
R2 -- User login \newline
R3 -- Start trip recording \newline
R4 -- Stop trip recording \newline
R5 -- Trip statistics computation \newline
R6 -- Trip history visualization \newline
R7 -- Trip weather enrichment
&
D1 -- GPS accuracy is sufficient to associate positions with map segments \newline
D4 -- External map and weather services return correct data \newline
D5 -- Users’ devices have intermittent or active network connectivity \newline
D10 -- Users have a compatible device with GPS and network access
\\
\hline
\end{tabular}
\caption{Goal G1 Mapping to Functional Requirements and Domain Assumptions}
\end{table}



% -------------------- G2 --------------------

\begin{table}[H]
\centering
\begin{tabular}{|p{7cm}|p{7cm}|}
\hline
\multicolumn{2}{|p{15cm}|}{\textbf{G2: Manual path information:} Registered users can insert or edit path segment statuses and obstacles and decide whether each contribution is publishable.} \\
\hline
\textbf{Mapped Functional Requirements} &
\textbf{Mapped Domain Assumptions / Dependencies} \\
\hline
R8 -- Manual path creation \newline
R9 -- Segment status assignment \newline
R10 -- Obstacle creation \newline
R11 -- Mark contribution as publishable or private \newline
R12 -- Edit manual contributions
&
D3 -- Users generally act in good faith and report path status honestly \newline
D6 -- A bike path is defined as a dedicated bike track or low-traffic road \newline
D7 -- Conflicting reports can be meaningfully merged \newline
D10 -- Users have a reliable internet connection
\\
\hline
\end{tabular}
\caption{Goal G2 Mapping to Functional Requirements and Domain Assumptions}
\end{table}


% -------------------- G3 --------------------

\begin{table}[H]
\centering
\begin{tabular}{|p{7cm}|p{7cm}|}
\hline
\multicolumn{2}{|p{15cm}|}{\textbf{G3: Automated path information:} Registered users can enable sensor-assisted acquisition while biking; BBP detects candidate issues that must be confirmed or corrected by the user before publication.} \\
\hline
\textbf{Mapped Functional Requirements} &
\textbf{Mapped Domain Assumptions / Dependencies} \\
\hline
R13 -- Enable/disable automatic mode \newline
R14 -- Sensor data logging \newline
R15 -- Detection of candidate obstacles \newline
R16 -- Candidate list presentation \newline
R17 -- User confirmation, rejection, or correction \newline
R18 -- Conversion of confirmed candidates into reports
&
D2 -- Device sensors are calibrated and reliable enough to detect pothole-like events \newline
D5 -- Devices have sufficient computational capability and battery \newline
D8 -- Users explicitly enable automatic sensor-based data collection \newline
D10 -- Users have compatible devices and OS support
\\
\hline
\end{tabular}
\caption{Goal G3 Mapping to Functional Requirements and Domain Assumptions}
\end{table}

% -------------------- G4 --------------------
\begin{table}[H]
\centering
\begin{tabular}{|p{7cm}|p{7cm}|}
\hline
\multicolumn{2}{|p{15cm}|}{\textbf{G4: Route search and visualization:} Any user (registered or guest) can request bike paths between origin and destination and visualize one or more route options.} \\
\hline
\textbf{Mapped Functional Requirements} &
\textbf{Mapped Domain Assumptions / Dependencies} \\
\hline
R19 -- Public path search \newline
R20 -- Route computation \newline
R22 -- Ordered path list \newline
R23 -- Path visualization
&
D4 -- External map and routing services are available \newline
D6 -- Bike paths are correctly represented in map provider data \newline
D10 -- Users have a working internet connection
\\
\hline
\end{tabular}
\caption{Goal G4 Mapping to Functional Requirements and Domain Assumptions}
\end{table}

% -------------------- G5 --------------------
\begin{table}[H]
\centering
\begin{tabular}{|p{7cm}|p{7cm}|}
\hline
\multicolumn{2}{|p{15cm}|}{\textbf{G5: Path scoring:} BBP orders multiple route options by a path score combining segment quality and route effectiveness.} \\
\hline
\textbf{Mapped Functional Requirements} &
\textbf{Mapped Domain Assumptions / Dependencies} \\
\hline
R21 -- Path scoring
&
D7 -- A meaningful consolidated status can be derived from reports \newline
D4 -- Map data and segment metadata are accurate \newline
D10 -- External services respond within acceptable time
\\
\hline
\end{tabular}
\caption{Goal G5 Mapping to Functional Requirements and Domain Assumptions}
\end{table}

% -------------------- G6 --------------------
\begin{table}[H]
\centering
\begin{tabular}{|p{7cm}|p{7cm}|}
\hline
\multicolumn{2}{|p{15cm}|}{\textbf{G6: Merging:} BBP merges publishable information from multiple users about the same segments considering freshness and confirmations or contradictions.} \\
\hline
\textbf{Mapped Functional Requirements} &
\textbf{Mapped Domain Assumptions / Dependencies} \\
\hline
R24 -- Segment report storage \newline
R25 -- Periodic merging \newline
R26 -- Freshness handling \newline
R27 -- Majority handling
&
D3 -- Users act in good faith when submitting reports \newline
D7 -- Time- and majority-based rules produce reliable consolidated status \newline
D10 -- Backend infrastructure supports scheduled jobs
\\
\hline
\end{tabular}
\caption{Goal G6 Mapping to Functional Requirements and Domain Assumptions}
\end{table}

% -------------------- G7 --------------------
\begin{table}[H]
\centering
\begin{tabular}{|p{7cm}|p{7cm}|}
\hline
\multicolumn{2}{|p{15cm}|}{\textbf{G7: Privacy and compliance:} Personal and location data are protected and publication is always under user control.} \\
\hline
\textbf{Mapped Functional Requirements} &
\textbf{Mapped Domain Assumptions / Dependencies} \\
\hline
R11 -- Publishability control \newline
R17 -- User confirmation before publication \newline
R29 -- Data removal by administrators
&
Regulatory constraint -- GDPR or equivalent data protection regulations apply \newline
D8 -- Users explicitly consent to sensor usage \newline
D10 -- Secure communication channels are available
\\
\hline
\end{tabular}
\caption{Goal G7 Mapping to Functional Requirements and Domain Assumptions}
\end{table}

% -------------------- G8 --------------------
\begin{table}[H]
\centering
\begin{tabular}{|p{7cm}|p{7cm}|}
\hline
\multicolumn{2}{|p{15cm}|}{\textbf{G8: Weather \& context enrichment:} BBP enriches trip data with meteorological information retrieved from an external service.} \\
\hline
\textbf{Mapped Functional Requirements} &
\textbf{Mapped Domain Assumptions / Dependencies} \\
\hline
R7 -- Trip weather enrichment
&
D4 -- Weather service is available and returns correct data \newline
D5 -- Network connectivity exists at least intermittently \newline
D10 -- External APIs respond within acceptable time
\\
\hline
\end{tabular}
\caption{Goal G8 Mapping to Functional Requirements and Domain Assumptions}
\end{table}


\subsection{Performance Requirements}
\begin{enumerate}
    \item The system should return a list of candidate paths between origin and destination within 10 seconds for typical urban distances (up to ~15 km) under normal load.
    \item Trip recording must sample GPS at least every 5 seconds when in motion.
    \item Sensor sampling should be frequent enough (e.g., $\geq 10$ Hz) to detect pothole-like events, but not so frequent as to drain the battery excessively.
    \item Under normal conditions, the system should support concurrent use by several thousand users without noticeable degradation of response time for basic operations (trip list, path search).

\end{enumerate}

\subsection{Design Constraints}
\subsubsection{Standards Compliance}
\begin{enumerate}
    \item \textbf{Data protection:} GDPR or analogous regulations in target regions.
    \item \textbf{Security:} Recommended best practices (e.g., OWASP for web/mobile applications).
    \item \textbf{Map licensing:} Comply with provider’s license (e.g., attribution, usage limits).
\end{enumerate}

\subsubsection{Hardware Limitations}
\begin{enumerate}
    \item BBP is primarily used on smartphones; performance and UI must accommodate mid-range devices (limited CPU, battery and memory).
    \item GPS and sensors may be temporarily unavailable or degraded (e.g., tunnels, urban canyons).
\end{enumerate}

\subsubsection{Any Other Constraint}
\begin{enumerate}
    \item Application requires a reasonably recent browser/OS with support for HTTPS and JavaScript.
    \item Offline usage: full trip recording should work with intermittent network; upload and path search may require connectivity.
\end{enumerate}

\subsection{Software System Attributes}
\subsubsection{Reliiability}
\begin{enumerate}
    \item Trips and path reports should be stored durably; server uses replicated storage and regular backups.
    \item Failure during recording (e.g., app crash) should not lose all data; partial tracks should be salvaged when possible.
\end{enumerate}

\subsubsection{Availablity}
\begin{enumerate}
    \item  BBP should target at least 99\% uptime.

    \item Core services (trip storage and path search) should be deployed with redundancy.
\end{enumerate}

\subsubsection{Security}
\begin{enumerate}
    \item All communications are over HTTPS.
    \item Passwords stored securely or external identity provider used.
    \item Access control: only the owner (and admins) can view private trips; only publishable info is visible to others.
    \item Rate limiting and request validation to reduce abuse.
\end{enumerate}
\subsubsection{Maintainability}
\begin{enumerate}
    \item Modular design separating front-end, back-end APIs and integration with external services.

    \item Well-documented public APIs and clear separation between trip logging, path search and merging modules.

\end{enumerate}

\subsubsection{Portability}
\begin{enumerate}   
    \item Front-end should run on major mobile OSes and modern browsers.


    \item Back-end deployable on common cloud infrastructures using containerization.

\end{enumerate}